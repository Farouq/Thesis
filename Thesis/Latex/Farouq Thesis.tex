\documentclass{uofsthesis-cs}

\usepackage{lipsum}% http://ctan.org/pkg/lipsum
\usepackage{color}

\usepackage{titlesec}
\usepackage{hyperref}
\usepackage{graphicx}
\usepackage{listings}
\usepackage{xcolor}

\usepackage{titling}

\usepackage{amsmath}

\usepackage{color}

\usepackage{float}

\usepackage[T1]{fontenc}
\usepackage[scaled]{beramono}

\usepackage{amssymb}

\definecolor{bluekeywords}{rgb}{0.13,0.13,1}
\definecolor{greencomments}{rgb}{0,0.5,0}
\definecolor{redstrings}{rgb}{0.9,0,0}


% Documentation for the uofsthesis-cs class is given in uofsthesis-cs.dvi
% 
% It is recommended that you read the CGSR thesis preparation
% guidelines before proceeding.
% They can be found at http://www.usask.ca/cgsr/thesis/index.htm

%%%%%%%%%%%%%%%%%%%%%%%%%%%%%%%%%%%%%%%%%%%%%%%%%%%%%%%%%%%%%%%%%%%%%%%%%%%%%%
% FRONTMATTER - In this section, specify information to be used to
% typeset the thesis frontmatter.
%%%%%%%%%%%%%%%%%%%%%%%%%%%%%%%%%%%%%%%%%%%%%%%%%%%%%%%%%%%%%%%%%%%%%%%%%%%%%%

% THESIS TITLE
% Specify the title. Set the capitalization how you want it.
\title{Semantic clone detection and benchmarking}
%\title{Using Ontology and Intermediate Language to Detect Semantic and Cross Language Clones}

% AUTHOR'S NAME
% Your name goes here.
\author{Farouq Al-omari}

% DEGREE SOUGHT.  
% Use \MSc or \PhD here
\degree{\PhD}         

% THESIS DEFENCE DATE
% Should be month/year, e.g. July 2004
\defencedate{December/2017}


% NAME OF ACADEMIC UNIT
%
% The following two commands allow you to specify the academic unit you belong to.
% This will appear on the title page as
% ``<academic unit> of <department>''.
% So if you are in the division of biomedical engineering you would need to do:
% \department{Biomedical Engineering}
% \academicunit{Division}
%
% The default is ``Department of Computer Science'' if these commands
% are not given.
%
% If you are in a discipline other than Computer Science, uncomment the following line and
% specify your discipline/department.  Default is 'Computer Science'.
% \department{If not Computer Science, put the name of your department here}

% If you are not in a department, but say, a division, uncomment the following line.
% \academicunit{Put the type of academic unit you belong to here, e.g. Division, College}


% PERMISSION TO USE ADDRESS
%
% If you are not in Comptuer Science you will want to change the
% address on the Permission to Use page.  This is done using the
% \ptuaddress{}.  Example:
%
% \ptuaddress{Head of the Department of Computer Science\\
% 176 Thorvaldson Building\\
% 110 Science Place\\
% University of Saskatchewan\\
% Saskatoon, Saskatchewan\\
% Canada\\
% S7N 5C9
% }

% ABSTRACT
\abstract{
	
Developers copy and paste their code to speed up the development process. Some-times, they copy code from an open source system and reuse it with or without modifications. The resulted similar or identical code fragments in terms of syntax or semantic are called code clones. Software cloning researches indicate the existence of code clones in all software systems; on average 5\% to 20\% of software code is cloned. Due to clones impact, either positive or negative impacts, its important to locate, track and manage them in the source code. Many techniques for detecting code clones have been proposed. However, a recent study shows that existing techniques have a limitation in detecting semantic clones. In this research, we proposed different techniques using intermediate language and ontology to detect clones across programming languages and semantic clones. We evaluated the efficiency of our techniques and compare it to start-of-art detection tools. Finally, we proposed a methodology to create a functional clone oracle that help researcher in this area to evaluate their tools.   
}

% THESIS ACKNOWLEDGEMENTS -- This can be free-form.
\acknowledgements{
Acknowledgements go here.  Typically you would at least thank your supervisor.
}

% THESIS DEDICATION -- Also free-form.  If you don't want a dedication, comment out the following
% line.
\dedication{This is the thesis dedication (optional)}

% LIST OF ABBREVIATIONS - Sample  
% If you don't want a list of abbreviations, comment the following 4 lines.
\loa{\abbrev{SCUBA}{Self Contained Underwater Breathing Apparatus}
\abbrev{LOF}{List of Figures}
\abbrev{LOT}{List of Tables}
}

%%%%%%%%%%%%%%%%%%%%%%%%%%%%%%%%%%%%%%%%%%%%%%%%%%%%%%%%%%%%%%%%
% END OF FRONTMATTER SECTION
%%%%%%%%%%%%%%%%%%%%%%%%%%%%%%%%%%%%%%%%%%%%%%%%%%%%%%%%%%%%%%%%

\begin{document}

% Typeset the title page
\maketitle

% Typeset the frontmatter.  
\frontmatter

%%%%%%%%%%%%%%%%%%%%%%%%%%%%%%%%%%%%%%%%%%%%%%%%%%%%%%%%%%%%%%%%
% FIRST CHAPTER OF THESIS BEGINS HERE
%%%%%%%%%%%%%%%%%%%%%%%%%%%%%%%%%%%%%%%%%%%%%%%%%%%%%%%%%%%%%%%%

\chapter{Introduction}

\section{Background}

Software clones are defined as similar (near-miss) or identical (identical clones) code fragments in terms of syntax or semantic. Usually, these code fragments result from the practice of programmers copying and pasting which produces identical clones. However, if the copied code fragments have minor modifications, they result in near- miss clones. The case of having major modifications to these clones will result in their disappearance. Conversely, some clones are unintentionally introduced into software systems due to the programmer practice to achieve common tasks or due to the use of library or API to implement common tasks. When code fragments have the same functionality, regardless of their syntactic, they called semantic clones. 
%"due to the programmer practice to achieve common tasks or due to the use of library or API to implement common tasks" sounds a bit confusing but maybe I just don't understand computers?

Software code cloning offers benefits during the development process. Usually, developers reuse their own code to save the time of rewriting, or they reuse others code to overcome some programming and design limitations\cite{Roy2007}. In some cases, the cloned code might have a serious problem; i.e. bugs that need more testing or updates in the maintenance phase \cite{Geiger2006,Rajapakse2005}. On the other hand, skilled developers pay more attention in order to choose higher quality, well tested, and bug free code to clone \cite{Cordy2003,Kim2005}. Practitioners have two different opinions about whether clones are harmful \cite{Bazrafshan2013, Ducasse1999, Kapser2006, Kamiya2002, Livieri2007,Mondal2012} or not \cite{Kapser2008,Cordy2003,Kapser2006,Kapser2004,Gode2011}. As a result, some studies target software clone harmfulness/usefulness \cite{Hordijk2009}. For example, \cite{Kim2005,RetoGeiger,Lozano2007} compared the co-changes of cloned to non-cloned code. Other studies compared the stability of cloned and non-cloned code \cite{Hotta2012,Harder2013,Mondal2012a,Krinke2008}. 



%importance of clone detection:
Over the decades, practitioners have proposed different techniques to detect both syntactic and semantic clones. Detecting syntactic clone is easier than semantic clones. In syntactic clone, the source code is normalized then transformed into other representations (token, tree, or vectors) before it used for comparison. However, in semantic clone detection more normalization needs to be done. For instance, dependencies and relationships (PDG) have to be identified and represented, and the functionality should be captured and used in comparing code units. A recent study shows that existing techniques and tools have some limitation in the detection of semantic clones (functional clones)~\cite{Wagner2016}.   

% importance of semantic clone detection
%importance of cross language clone detection

%Identifying cloned fragments have been the targeted by practitioner for a long time. 

%clone definition

More recently there has been an ongoing trend towards multi-language software development to take advantage of different programming languages~\cite{Kontogiannis2006}; specifically in the .NET context. For multi-language development, two key usage scenarios can be distinguished: (1) combining different programming languages within a single, often large and complex system, and (2) the use of several languages for re-implementation of a current system to support new client, application, or due to non-technical reasons. As a result, the ability to detect and manage similar code reuse patterns that might exist in these multiple languages systems becomes essential. While many clone detection tools are capable of supporting different programming languages, they lack actual cross-language support during detection time. Consequently, these tools only detect clones in one program language at the time, and do not detect clones that span over multiple programming languages.
%this sentence is messy: "the use of several languages for re-implementation of a current system to support new client, application, or due to non-technical reasons". The tense is off and it sounds like you're describing too many things in one sentence. 

%Semantically equivalent function clones: Similarity between the functions is based on the semantic similarity of the functions. Two functions with identi- cal functionality may be considered as clones even if they differ in names, have different order and names of arguments, and different names of local variables. This is like the Type IV clones that we provide in Section 7.2.  Fanta and Rajlich [78] ..... Roy 

The accuracy of both emerging and proposed techniques are needed to be evaluated in detecting all types of clones. Three major techniques are used for evaluation: (1) manual inspection of reported clones to identify true positives and false positives, (2) injecting the source code with artificially generated clones to measure how many clones the tool(s) are able to detect~\cite{Svajlenko2014,Roy2009a}, and 3) using benchmarks, for already identified and known clones in the system~\cite{Krutz2014,Bellon2007}.   

The rest of the proposal is organized as follows: Background topics of my research that includes clone definition, intermediate language, Ontology, ontology matching, and matching algorithms which are presented in section 2. Section 3 presents the related work to our research. Section 4 states the thesis statement. Section 5 presents our study in clone detection across programming languages. Section 6 describes a proposed study in detecting semantic clones. Another proposed technique in the detection of semantic clones is presented in section 7. In section 8, we describe a technique to build a clone oracle. Finally, Section 9 concludes the paper. 

\input{2-Background.tex}



% Since thesis chapters are very long and there are a lot of them, it is recommended
% that you put each chapter in a separate .tex file and \input each one of them
% in order.  For example:
%
% \input chap1.tex
% \input chap2.tex
% ...
%
% The \input command inserts contents of the specified file at the point of the command.

%%%%%%%%%%%%%%%%%%%%%%%%%%%%%%%%%%%%%%%%%%%%%%%%%%%%%%%%%%%%%%%
% SUBSEQUENT CHAPTERS (or \input's)  GO HERE
%%%%%%%%%%%%%%%%%%%%%%%%%%%%%%%%%%%%%%%%%%%%%%%%%%%%%%%%%%%%%%%






%%%%%%%%%%%%%%%%%%%%%%%%%%%%%%%%%%%%%%%%%%%%%%%%%%%%%%%%%%%%%%%%
% The Bibliograpy should go here. BEFORE appendices!
%%%%%%%%%%%%%%%%%%%%%%%%%%%%%%%%%%%%%%%%%%%%%%%%%%%%%%%%%%%%%%%%


% Typeset the Bibliography.  The bibliography style used is "plain".
% Optionally, you can specify the bibliography style to use:
% \uofsbibliography[stylename]{yourbibfile}

%


%\bibliographystyle{plain}

\uofsbibliography{bibliography}
% If you are not using bibtex, comment the line above and uncomment
% the line below.  
%Follow the line below with a thebibliography environmentand bibitems.  
% Note: use of bibtex is usually the preferred method.

%\uofsbibliographynobibtex


%%%%%%%%%%%%%%%%%%%%%%%%%%%%%%%%%%%%%%%%%%%%%%%%%%%%%%%%%%%%%%%%%%%%%%%%%
% APPENDICES
%
% Any chapters appearing after the \appendix command get numbered with
% capital letters starting with appendix 'A'.
% New chapters from here on will be called 'Appendix A', 'Appendix B'
% as opposed to 'Chapter 1', 'Chapter 2', etc.
%%%%%%%%%%%%%%%%%%%%%%%%%%%%%%%%%%%%%%%%%%%%%%%%%%%%%%%%%%%%%%%%%%%%%%%%%%

% Activate thesis appendix mode.
\uofsappendix

% Put appendix chapters in the appendices environment so that they appear correcty
% in the table of contents.  You can use \input's here as well.
\begin{appendices}

\chapter{Sample Appendix}

Stuff for this appendix goes here.

\chapter{Another Sample Appendix}

Stuff for this appendix goes here.

\end{appendices}


\end{document}
